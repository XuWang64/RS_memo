\documentclass[aps,tightenlines,16pt]{ctexart}

\usepackage{slashed}
\usepackage{amsmath,amsfonts,amssymb}
% \usepackage{bm}  %黑体希腊字母
\usepackage{bbm} %空心字母数字
\usepackage{ctex}
\usepackage{float}
%\usepackage[dvips]{graphicx}
\usepackage[english]{babel}
\allowdisplaybreaks[4]  %公式环境换行
\numberwithin{equation}{section}
\usepackage{slashed} %费曼slashed
\usepackage{simplewick} %wick收缩
\usepackage[left=2cm,right=2cm,top=2.54cm,bottom=2.54cm]{geometry}
%\usepackage[dvipdfm,pdfstartview=FitH]{hyperref}
%\usepackage{cancel}
%\usepackage{forest}
%\usepackage{leftidx}  %设置上下标
\usepackage{cite}
\usepackage[justification=centering]{caption}
\usepackage{graphicx, subfigure}
\usepackage{indentfirst}  %段落首行缩进
\usepackage{hyperref}  %设定引用公式跳转链接
\usepackage{color}  %设置字体颜色
\usepackage{tikz,pgf}
%\usepackage{tikz-feynman} 
%\tikzfeynmanset{compat=1.1.0}
\usepackage{braket}
%\usepackage{txfonts}  %平行符号
%\usepackage{fancyhdr}  %左偶右奇
\usepackage{multirow}
\usepackage{booktabs}
\usepackage{cancel}   
\usepackage{threeparttable}  
\usepackage{diagbox}
\usepackage{extpfeil}  %等号上加文字
\usepackage{extarrows}

\usetikzlibrary{calc}
%\usetikzlibrary{arrows.meta}
\usetikzlibrary{intersections}
\usetikzlibrary{trees}
\usetikzlibrary{decorations.pathmorphing}
\usetikzlibrary{decorations.markings}
\usetikzlibrary{patterns}
\tikzset{
   global scale/.style={
      scale=#1,
      every node/.append style={scale=#1}},
   photon/.style={decorate, decoration={snake}, draw=red},
   nucleon/.style={draw=black, postaction={decorate},
      decoration={markings,mark=at position .55 with{\arrow[draw=black]{>}}}},
   pion/.style={draw=blue, postaction={decorate},
      decoration={markings,mark=at position .55 with{\arrow[draw=blue]{}}}},
    }


\newcommand{\bl}{\boldsymbol{l}}
\newcommand{\bk}{\boldsymbol{k}}
\newcommand{\bp}{\boldsymbol{p}}
\newcommand{\bP}{\boldsymbol{P}}
\newcommand{\bq}{\boldsymbol{q}}
\newcommand{\bA}{\boldsymbol{A}}
\newcommand{\bM}{\boldsymbol{M}}
\newcommand{\bV}{\boldsymbol{V}}
\newcommand{\ba}{\boldsymbol{a}}
\newcommand{\bb}{\boldsymbol{b}}
\newcommand{\bx}{\boldsymbol{x}}
\newcommand{\bep}{\boldsymbol{\epsilon}}
\newcommand{\bsi}{\boldsymbol{\sigma}}
\newcommand{\bL}{\boldsymbol{L}}
\newcommand{\bJ}{\boldsymbol{J}}
\newcommand{\br}{\boldsymbol{r}}
\newcommand{\bs}{\boldsymbol{s}}
\newcommand{\bS}{\boldsymbol{S}}
\newcommand{\bi}{\boldsymbol{i}}
\newcommand{\bI}{\boldsymbol{I}}
\newcommand{\bB}{\boldsymbol{B}}  
\newcommand{\sP}{\slashed{P}} 
\newcommand{\spp}{\slashed{p}} 
\newcommand{\sk}{\slashed{k}} 
\newcommand{\sq}{\slashed{q}}
\newcommand{\sD}{\slashed{D}} 
\newcommand{\sA}{\slashed{A}}
\newcommand{\sep}{\slashed{\epsilon}} 
\newcommand{\spar}{\slashed{\partial}} 
\newcommand{\Pmu}{P^\mu} 
\newcommand{\pmu}{p^\mu} 
\newcommand{\kmu}{k^\mu} 
\newcommand{\qmu}{q^\mu}
\newcommand{\gmu}{\gamma^\mu}
\newcommand{\bpi}{\boldsymbol{\pi}}
\newcommand{\btau}{\boldsymbol{\tau}}
\newcommand{\brho}{\boldsymbol{\rho}}
\newcommand{\md}{\mathrm{d}}
\newcommand{\mB}{\mathbf{B}}
\newcommand{\mO}{\mathcal{O}}
\newcommand{\mL}{\mathcal{L}}
\newcommand{\bm}[1]{\mbox{\boldmath{$#1$}}}
\newcommand{\Tr}{\text{Tr}}
\allowdisplaybreaks


\begin{document}\large
     %\title{手征微扰场论阅读笔记} 
     \title{手征微扰场论}
     
\renewcommand{\today}{\number\year 年 \number\month 月 \number\day 日}
 \author{王旭}
 \maketitle
 %\newpage
 \setlength{\parindent}{2em}  %首行缩进两个中文字符
 \hypersetup{hypertex=true,
            colorlinks=true,
            linkcolor=blue,
            anchorcolor=blue,
            citecolor=blue}  %设定引用公式跳转链接
 \renewcommand\thesubsection{\arabic {subsection}}
 \renewcommand\contentsname{目录}
\tableofcontents
\newpage 
\setcounter{page}{1}

\section{$\pi\pi$散射的RS方程}

\begin{enumerate}
   \item $\pi\pi$散射:
   Roy在研究$\pi \pi$散射时提出RS方程。RS方程是一种特殊的色散关系。这样的色散关系中没有任何不确定的减除常数,不过代价是耦合了所有的分波。
   在$\pi\pi$散射过程中,首先就是改写固定$t$的两次减除色散关系,将左右割线的贡献用右手割线表示,接着利用交叉对称性可以将减除常数用阈值处的振幅表示,最后对$t$做分波展开就可以得到RS方程。

   首先考虑一个固定$t$的两次减除色散关系:
   \begin{align}\label{T}
       \begin{aligned}
       T(s,t,u)=&\alpha(t)+s\beta(t)+\frac{s^2}{\pi}\int_{4m_{\pi}^2}^{\infty}\mbox{d}s^{\prime}\frac{\mathrm{Im}T(s^{\prime},t,u^{\prime})}{s^{\prime 2}(s^{\prime}-s)}\\
       &+\frac{s^2}{\pi}\int_{-\infty}^{-t}\mbox{d}s^{\prime}\frac{\mathrm{Im}T(s^{\prime},t,u^{\prime})}{s^{\prime 2}(s^{\prime}-s)},
       \end{aligned}
   \end{align}
   
\end{enumerate}



第一步就是改写色散关系,将(\ref{T})中$u$道的运动学因子利用
\begin{align*}
   \begin{aligned}
   \frac{s^2}{s^{\prime 2}(s^{\prime}-s)}=&-\frac{u^2}{u^{\prime 2}(u^{\prime}-u)}-\Bigl[\frac{1}{u^{\prime}}+\frac{1}{4m_{\pi}^2-t-u^{\prime}}+\frac{4m_{\pi}^2-t}{u^{\prime 2}}\Bigr]\\
   &-s\Bigl[\frac{1}{(4m_{\pi}^2-t-u^{\prime})^2}-\frac{1}{u^{\prime 2}}\Bigr]
   \end{aligned}
\end{align*}























\newpage
\bibliographystyle{unsrt}
\bibliography{RS}






\end{document}